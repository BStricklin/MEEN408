\documentclass{article}
\usepackage{listings} 				%for bash and c++ code
%\usepackage{relsize}
\usepackage{color,soul}
\usepackage{graphicx}
\usepackage[justification=centering,singlelinecheck=false]{caption}
\usepackage{setspace}
\usepackage[margin=1in]{geometry}
\usepackage[dvipsnames]{xcolor}
\usepackage{amssymb}				% enables formulas
\usepackage{amsmath}				% enables formulas-
\usepackage{fancyhdr}
\usepackage{array}
\usepackage{gensymb}				% enables symbols
\usepackage{lastpage}
\usepackage{textcomp} 				% enables copyright
\usepackage{booktabs}
\usepackage{pdfpages}
\usepackage{tabto}					% enables tabbing
\usepackage{hyperref}				% enables hyperlinks
\usepackage[dvipsnames]{xcolor}
\hypersetup{
	colorlinks,
	linkcolor={blue!50!black},
	citecolor={blue!50!black},
	urlcolor={blue!50!black},
	linktoc=page
}

%% Figure Placement Options
\usepackage{chngcntr}				% enables counter
\counterwithin{table}{subsection}		% enables table pre-labels
\counterwithin{figure}{subsection}		% enables figure pre-labels
%\setcounter{secnumdepth}{2}			% sets counter recognitions to sssection for label capability
\makeatletter
\renewcommand*\l@figure{\@dottedtocline{1}{1em}{3.2em}}
\makeatother
\usepackage{placeins}
\usepackage{subcaption}
\let\Oldsection\section
\renewcommand{\section}{\FloatBarrier\Oldsection}
\let\Oldsubsection\subsection
\renewcommand{\subsection}{\FloatBarrier\Oldsubsection}
\let\Oldsubsubsection\subsubsection
\renewcommand{\subsubsection}{\FloatBarrier\Oldsubsubsection}

%% Table of Contents Setup (clickable)
%\usepackage[hidelinks=true]{hyperref}
\renewcommand{\contentsname}{Table of Contents}

%% Table Setup
\newcolumntype{A}{ >{\centering\arraybackslash} m{2cm} }
\newcolumntype{B}{ >{\centering\arraybackslash} m{2cm} }
\newcolumntype{C}{ >{\centering\arraybackslash} m{6.5cm} }
\newcolumntype{D}{ >{\centering\arraybackslash} m{0.5\textwidth} }
\newcolumntype{E}{ >{\centering\arraybackslash} m{0.2\textwidth} }
\newcolumntype{F}{ >{\arraybackslash} m{0.3\textwidth} }
\newcolumntype{G}{ >{\arraybackslash} m{0.33\textwidth} }

%% no indentation
\setlength{\parindent}{0pt}

%%header and footer
\fancyhf{}
\renewcommand{\footrulewidth}{0.4pt}
\rhead{MEEN 408 Lab Manual}
\rfoot{Page~\textbf{\thepage}~of~\textbf{\pageref*{LastPage}}}
\lfoot{Introduction to Robotics, Texas A\&M University}
\pagestyle{fancy}

%%  Defining Styles for Code
\definecolor{listinggray}{gray}{0.9}
\definecolor{lbcolor}{rgb}{0.95,0.95,0.95}
\lstdefinestyle{numbers} {numbers=left, stepnumber=1, numberstyle=\tiny, numbersep=10pt}
\lstdefinestyle{MyFrame}{backgroundcolor=\color{gray},frame=shadowbox}
\lstdefinestyle{C++Style} 
		{	language=[GNU]C++,
			style=numbers,
			backgroundcolor=\color{lbcolor},
			columns=fixed,
			showstringspaces=false,
			breaklines=true,
			showtabs=false,
			showspaces=false, 
			%style=MyFrame, 
			basicstyle=\ttfamily, 
			keywordstyle=\color{blue}\ttfamily, 
			stringstyle=\color{ForestGreen}\ttfamily,	
			commentstyle=\color{Dandelion}\ttfamily,
			identifierstyle=\ttfamily,
			morecomment=[l][\color{magenta}]{\#}
		}
		
\def\CC{{C\nolinebreak[4]\hspace{-.05em}\raisebox{.4ex}{\tiny\bf ++}}}


\begin{document}
%%Front Page
%% Front Page
{\setstretch{2}
\thispagestyle{empty}
\begin{center}
\textbf{\Huge MEEN 408: \\ Introduction to Robotics} \\
\Huge Lab Manual \\
\end{center}
}

\setstretch{1}
\clearpage

%% Revision Table
\begin{center}
	\begin{table}
	\caption{Revision History}
	\label{tab:Revision History} 
	\begin{tabular}{ A B A B C}
		\toprule
		\textbf{Version} 	& \textbf{Date} 		& \textbf{Author} 	& 	\textbf{Reviewed} 	& \textbf{Details} \\ \midrule
		1.0					& 11/24/2016 			& AJE 				& 	---			 		& Document Creation \\ \midrule
							& 						&					&			 			& \\ \midrule
							& 						&					&			 			& \\ \bottomrule

	\end{tabular}
	\end{table}
\end{center}
\clearpage

%% Table of Contents
\tableofcontents

\clearpage
%% List of Figures and List of Tables
\listoffigures
\listoftables

\clearpage
%% Acronyms and Abbreviations+
\begin{center}
	\begin{table}
		\centering
		\caption{Acronyms and Abbreviations}
		\label{tab:Acronyms and Abbreviations} 
		\begin{tabular}{ D D }
			\toprule
			\textbf{Acronym or Abbreviation} 	& \textbf{Meaning}  						\\ \midrule
			ROS									& Robot Operating System					\\ \midrule
			SSH 								& Secure SHell			 					\\ \bottomrule
		\end{tabular}
	\end{table}
\end{center}

\clearpage
%% Some basic info for whomsoever reads this document
\section*{A Note to The Reader}
\addcontentsline{toc}{section}{A Note to The Reader}
\label{sec:a_note}
\subsection*{Digital Document}
This document was prepared in \LaTeX -- a dynamic document processor. As such, the document is intended for digital use. Throughout, there are hyperlinks, highlighted in blue, like \href{https://en.wikipedia.org/wiki/Robot}{this}.
%% End of Preamble
\clearpage
\section{Lab 1: ROS Installation and \protect\CC}
\clearpage
\section{Lab 2: Hello World with ROS}
\clearpage
\section{Lab 3: Topic Example}
\clearpage
\section{Lab 4: Control of Built-In LEDS}
\lstinputlisting[style=C++Style]{"../LabCode/Lab04\string_LedsCPP/leds.cpp"}

\clearpage
\section{Lab 5: GPIO and using \protect\CC}
\lstinputlisting[style=C++Style]{"../LabCode/Lab05\string_GPIOCPP/gpio.cpp"}

\clearpage
\section{Lab 6: ADC and \protect\CC}
\lstinputlisting[style=C++Style]{"../LabCode/Lab06\string_ADCCPP/adc.cpp"}

\clearpage
\section{Lab 7: PWM and \protect\CC}
\lstinputlisting[style=C++Style]{"../LabCode/Lab07\string_PWMCPP/pwm.cpp"}

\clearpage
\section{Lab 8: Quadrature Encoder}
	\subsection{ eQEP Setup}
	\subsection{ eQEP with \protect\CC}
	\lstinputlisting[style=C++Style]{"../LabCode/Lab08\string_EQEPCPP/quad.cpp"}
	
\clearpage
\section{Lab 9: Putting it all Together---Making your own \protect\CC ~Library}


\clearpage
\section{Lab 10: ROS Network Setting}

\clearpage
\section{Lab 11: Servomotor}

\clearpage
\section{Lab 12: DC Motor, BBB, and ROS}
\end{document}
