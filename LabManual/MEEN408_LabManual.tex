\documentclass{article}
\usepackage{listings} 				%for bash and c++ code
%\usepackage{relsize}
\usepackage{color,soul}
\usepackage{graphicx}
\usepackage[justification=centering,singlelinecheck=false]{caption}
\usepackage{setspace}
\usepackage[margin=1in]{geometry}
\usepackage[dvipsnames]{xcolor}
\usepackage{amssymb}				% enables formulas
\usepackage{amsmath}				% enables formulas-
\usepackage{fancyhdr}
\usepackage{array}
\usepackage{gensymb}				% enables symbols
\usepackage{lastpage}
\usepackage{textcomp} 				% enables copyright
\usepackage{booktabs}
\usepackage{pdfpages}
\usepackage{tabto}					% enables tabbing
\usepackage{hyperref}				% enables hyperlinks
\usepackage[dvipsnames]{xcolor}
\hypersetup{
	colorlinks,
	linkcolor={blue!50!black},
	citecolor={blue!50!black},
	urlcolor={blue!50!black},
	linktoc=page
}

%% Figure Placement Options
\usepackage{chngcntr}				% enables counter
\counterwithin{table}{subsection}		% enables table pre-labels
\counterwithin{figure}{subsection}		% enables figure pre-labels
%\setcounter{secnumdepth}{2}			% sets counter recognitions to sssection for label capability
\makeatletter
\renewcommand*\l@figure{\@dottedtocline{1}{1em}{3.2em}}
\makeatother
\usepackage{placeins}
\usepackage{subcaption}
\let\Oldsection\section
\renewcommand{\section}{\FloatBarrier\Oldsection}
\let\Oldsubsection\subsection
\renewcommand{\subsection}{\FloatBarrier\Oldsubsection}
\let\Oldsubsubsection\subsubsection
\renewcommand{\subsubsection}{\FloatBarrier\Oldsubsubsection}

%% Table of Contents Setup (clickable)
%\usepackage[hidelinks=true]{hyperref}
\renewcommand{\contentsname}{Table of Contents}

%% Table Setup
\newcolumntype{A}{ >{\centering\arraybackslash} m{2cm} }
\newcolumntype{B}{ >{\centering\arraybackslash} m{2cm} }
\newcolumntype{C}{ >{\centering\arraybackslash} m{6.5cm} }
\newcolumntype{D}{ >{\centering\arraybackslash} m{0.5\textwidth} }
\newcolumntype{E}{ >{\centering\arraybackslash} m{0.2\textwidth} }
\newcolumntype{F}{ >{\arraybackslash} m{0.3\textwidth} }
\newcolumntype{G}{ >{\arraybackslash} m{0.33\textwidth} }

%% no indentation
\setlength{\parindent}{0pt}

%%header and footer
\fancyhf{}
\renewcommand{\footrulewidth}{0.4pt}
\rhead{MEEN 408 Lab Manual}
\rfoot{Page~\textbf{\thepage}~of~\textbf{\pageref*{LastPage}}}
\lfoot{Introduction to Robotics, Texas A\&M University}
\pagestyle{fancy}

%%  Defining Styles for Code
\definecolor{listinggray}{gray}{0.9}
\definecolor{lbcolor}{rgb}{0.95,0.95,0.95}
\lstdefinestyle{numbers} {numbers=left, stepnumber=1, numberstyle=\tiny, numbersep=10pt}
\lstdefinestyle{MyFrame}{backgroundcolor=\color{gray},frame=shadowbox}
\lstdefinestyle{C++Style} 
		{	language=[GNU]C++,
			style=numbers,
			backgroundcolor=\color{lbcolor},
			columns=fixed,
			showstringspaces=false,
			breaklines=true,
			showtabs=false,
			showspaces=false, 
			%style=MyFrame, 
			basicstyle=\ttfamily, 
			keywordstyle=\color{blue}\ttfamily, 
			stringstyle=\color{ForestGreen}\ttfamily,	
			commentstyle=\color{Dandelion}\ttfamily,
			identifierstyle=\ttfamily,
			morecomment=[l][\color{magenta}]{\#}
		}
		
\def\CC{{C\nolinebreak[4]\hspace{-.05em}\raisebox{.4ex}{\tiny\bf ++}}}


\begin{document}
%%Front Page
%% Front Page
{\setstretch{2}
\thispagestyle{empty}
\begin{center}
\textbf{\Huge MEEN 408: \\ Introduction to Robotics} \\
\Huge Lab Manual \\
\end{center}
}

\setstretch{1}
\clearpage

%% Revision Table
\begin{center}
	\begin{table}
	\caption{Revision History}
	\label{tab:Revision History} 
	\begin{tabular}{ A B A B C}
		\toprule
		\textbf{Version} 	& \textbf{Date} 		& \textbf{Author} 	& 	\textbf{Reviewed} 	& \textbf{Details} \\ \midrule
		1.0					& 11/24/2016 			& AJE 				& 	---			 		& Document Creation \\ \midrule
							& 						&					&			 			& \\ \midrule
							& 						&					&			 			& \\ \bottomrule

	\end{tabular}
	\end{table}
\end{center}
\clearpage

%% Table of Contents
\tableofcontents

\clearpage
%% List of Figures and List of Tables
\listoffigures
\listoftables

\clearpage
%% Acronyms and Abbreviations+
\begin{center}
	\begin{table}
		\centering
		\caption{Acronyms and Abbreviations}
		\label{tab:Acronyms and Abbreviations} 
		\begin{tabular}{ D D }
			\toprule
			\textbf{Acronym or Abbreviation} 	& \textbf{Meaning}  						\\ \midrule
			ROS									& Robot Operating System					\\ \midrule
			SSH 								& Secure SHell			 					\\ \bottomrule
		\end{tabular}
	\end{table}
\end{center}

\clearpage
%% Some basic info for whomsoever reads this document
\section*{A Note to The Reader}
\addcontentsline{toc}{section}{A Note to The Reader}
\label{sec:a_note}
\subsection*{Digital Document}
This document was prepared in \LaTeX -- a dynamic document processor. As such, the document is intended for digital use. Throughout, there are hyperlinks, highlighted in blue, like \href{https://en.wikipedia.org/wiki/Robot}{this}.
%% End of Preamble
\clearpage
\section{Lab 1: ROS Installation and \protect\CC}
\clearpage
\section{Lab 2: Hello World with ROS}
\clearpage
\section{Lab 3: Topic Example}
\clearpage
\section{Lab 4: Control of Built-In LEDS}
\begin{lstlisting}[style=C++Style]
#include <stdio.h>
#include <unistd.h>
#include <iostream>
using namespace std;

int main() {
  cout << "LED Flash Start" << endl;

  FILE* LEDHandle = NULL;
  const char* LEDBrightness =
      "/sys/class/leds/beaglebone:green:usr3/brightness";
  for (int i = 0; i < 10; i++) {
    cout << "on " << i << endl;
    if ((LEDHandle = fopen(LEDBrightness, "r+")) != NULL) {
      fwrite("1", sizeof(char), 1, LEDHandle);
      fclose(LEDHandle);
    }
    sleep(1);
    cout << "off " << i << endl;
    if ((LEDHandle = fopen(LEDBrightness, "r+")) != NULL) {
      fwrite("0", sizeof(char), 1, LEDHandle);
      fclose(LEDHandle);
    }
    sleep(1);
  }
  cout << "LED Flash End" << endl;
  return 0;
}
\end{lstlisting}

\clearpage
\section{Lab 5: GPIO and using \protect\CC}
\begin{lstlisting}[style=C++Style]
#include <stdio.h>
#include <string.h>
#include <unistd.h>
#include <iostream>
#define MAX 64
using namespace std;

int flashGPIOLED(int, int);

int main() {
  // readButton(117)
  flashGPIOLED(60, 5);
  return 0;
}

int flashGPIOLED(int GPIOPin, int times) {
  cout << "GPIO LED Flash Pin: " << GPIOPin << " start." << endl;

  // Create Strings that point to the GPIO Pin Value and Direction files
  FILE* LEDHandle = NULL;
  char setValue[4];
  char GPIOString[4];
  char GPIOValue[MAX];
  char GPIODirection[MAX];
  sprintf(GPIOString, "%d", GPIOPin);
  sprintf(GPIOValue, "/sys/class/gpio/gpio%d/value", GPIOPin);
  sprintf(GPIODirection, "/sys/class/gpio/gpio%d/direction", GPIOPin);

  // Export the Pin Number (this will make the Pin directory we can then use)
  // First we see if it is possible to create the directory. If not, quit.
  if ((LEDHandle = fopen("/sys/class/gpio/export", "ab")) ==
      NULL) {  // note that this opens the file
    printf("Cannot export the GPIO Pin");
    return 1;
  }
  // If the above works, export the pin:
  strcpy(setValue, GPIOString);                   // copy pin number to setValue
  fwrite(&setValue, sizeof(char), 2, LEDHandle);  // write set Value to
                                                  // LEDHANDLE
  fclose(LEDHandle);                              // and close the file

  // set pin direction
  if ((LEDHandle = fopen(GPIODirection, "rb+")) == NULL) {
    printf("Cannot open direction handle. \n");
    return 1;
  }
  strcpy(setValue, "out");
  fwrite(&setValue, sizeof(char), 3, LEDHandle);
  fclose(LEDHandle);

  // flash the led on the gpio pin
  for (int i = 0; i < times * 2; i++) {
    if ((LEDHandle = fopen(GPIOValue, "rb+")) == NULL) {
      printf("Cannot open value handle. \n");
      return 1;
    }
    cout << &LEDHandle << endl;  // this just prints out the file pointer value.
                                 // not sure why this is here
    if (i % 2 == 1) {
      strcpy(setValue, "0");
    } else {
      strcpy(setValue, "1");
    }
    fwrite(&setValue, sizeof(char), 1, LEDHandle);
    fclose(LEDHandle);
    sleep(1);
  }

  if ((LEDHandle = fopen("/sys/class/gpio/unexport", "ab")) == NULL) {
    printf("Cannot unexport GPIO Pin.\n");
    return 1;
  }
  strcpy(setValue, GPIOString);
  fwrite(&setValue, sizeof(char), 2, LEDHandle);
  fclose(LEDHandle);

  cout << "GPIO LED Flash PIN: " << GPIOPin << " end" << endl;
  return 0;
}
\end{lstlisting}

\clearpage
\section{Lab 6: ADC and \protect\CC}
\begin{lstlisting}[style=C++Style]
#include <stdio.h>
#include <unistd.h>
#include <iostream>
#include <sstream>
#include <string>
using namespace std;

int main() {
  cout << "ADC Start" << endl;
  FILE* ADCHandler = NULL;
  const char* ADCVoltage = "/sys/bus/iio/devices/iio:device0/in_voltage5_raw";

  char ADCVoltageRead[5] = {0};
  int Voltage;

  while (1) {
    if ((ADCHandler = fopen(ADCVoltage, "r")) != NULL) {
      fread(ADCVoltageRead, sizeof(char), sizeof(ADCVoltageRead - 1),
            ADCHandler);
      fclose(ADCHandler);
      stringstream ss(ADCVoltageRead);
      ss >> Voltage;
      cout << Voltage << endl;
      //	printf("%s", ADCVoltageRead);
      usleep(50000);
    }
  }
  cout << "ADC End" << endl;
  return 0;
}
\end{lstlisting}

\clearpage
\section{Lab 7: PWM and \protect\CC}
\begin{lstlisting}[style=C++Style]
#include <stdio.h>
#include <unistd.h>
#include <iostream>
using namespace std;

#define PERIOD 1000000

int main() {
  cout << "PWM Start" << endl;

  FILE* PWMHandle = NULL;
  const char* PWMPeriod = "/sys/class/pwm/pwmchip0/pwm0/period";
  const char* PWMDutyCycle = "/sys/class/pwm/pwmchip0/pwm0/duty_cycle";
  const char* PWMEnable = "/sys/class/pwm/pwmchip0/pwm0/enable";
  char setValue[10];

  // Set PWM period, duty cycle, and enabled status
  if ((PWMHandle = fopen(PWMPeriod, "r+")) != NULL) {
    fwrite("1000000", sizeof(char), 7, PWMHandle);
    fclose(PWMHandle);
  }
  if ((PWMHandle = fopen(PWMDutyCycle, "r+")) != NULL) {
    fwrite("0", sizeof(char), 1, PWMHandle);
    fclose(PWMHandle);
    // cout << "DutyCycle " << sizeof(PERIOD/2)/sizeof(char) << endl;
  }
  if ((PWMHandle = fopen(PWMEnable, "r+")) != NULL) {
    fwrite("1", sizeof(char), 1, PWMHandle);
    fclose(PWMHandle);
  }

  // increase the duty cycle from 0% to 100% in 10 seconds smoothly
  double timeToFullLight = 10; //seconds
  int numberOfIncrements = 1000;
  for (int i = 0; i < numberOfIncrements; i++) {
    cout << "count " << i << endl;
    sprintf(setValue, "%d", int(PERIOD * i / numberOfIncrements));
    if ((PWMHandle = fopen(PWMDutyCycle, "r+")) != NULL) {
      fwrite(setValue, sizeof(char), sizeof(setValue), PWMHandle);
      fclose(PWMHandle);
    }
   usleep(int(timeToFullLight/numberOfIncrements*1000000));
  }

  // reset duty cycle to 0 and disable
  if ((PWMHandle = fopen(PWMDutyCycle, "r+")) != NULL) {
    fwrite("0", sizeof(char), 1, PWMHandle);
    fclose(PWMHandle);
    // cout << "DutyCycle " << sizeof(PERIOD/2)/sizeof(char) << endl;
  }
  if ((PWMHandle = fopen(PWMEnable, "r+")) != NULL) {
    fwrite("0", sizeof(char), 1, PWMHandle);
    fclose(PWMHandle);
  }

  cout << "PWM End" << endl;
  return 0;
}
\end{lstlisting}

\clearpage
\section{Lab 8: Quadrature Encoder}
	\subsection{ eQEP Setup}
	\subsection{ eQEP with \protect\CC}
	\begin{lstlisting}[style=C++Style]
	content...
	\end{lstlisting}
	
\clearpage
\section{Lab 9: Putting it all Together---Making your own \protect\CC ~Library}


\clearpage
\section{Lab 10: ROS Network Setting}
\begin{lstlisting}[style=C++Style]
content...
\end{lstlisting}

\clearpage
\section{Lab 11: Servomotor}
\begin{lstlisting}[style=C++Style]
content...
\end{lstlisting}

\clearpage
\section{Lab 12: DC Motor, BBB, and ROS}
\begin{lstlisting}[style=C++Style]
content...
\end{lstlisting}
\end{document}
